% Artículo de doble cara
\documentclass[12pt, oneside]{article}

%%%% Dependencias %%%%
% Imágenes de prueba
\usepackage{graphicx}
% Texto de prueba (lorem ipsum)
\usepackage{blindtext}
% Headers
\usepackage{fancyhdr}
% para condicionales
\usepackage{ifthen}
% Símbolos matemáticos
\usepackage[T1]{fontenc} 
% Para establecer los márgenes de la página
\usepackage[letterpaper, bindingoffset=0.5cm, left=3cm, right=2cm, top=2.5cm, bottom=2.5cm]{geometry}
% For times new roman
\usepackage{Times}

%%%% Declaraciones %%%%
% Inicia contador de páginas en cero
\setcounter{page}{0}

% Tamaño de la fuente para títulos
\font\titles=cmr12 at 18pt

\newcommand{\code}{215472278}
\newcommand{\calendar}{2021A}
\newcommand{\nrc}{103844}
\newcommand{\signSec}{D01}
\newcommand{\profName}{Profa. Becerra Velázquez Violeta del Rocío}

% Título
\newcommand{\actTitle}{FCFS}
\newcommand{\actNum}{ACTIVIDAD DE APRENDIZAJE 6: }
\title{\tiText}

% Macro con mi nombre
\newcommand{\myName}{Cristian Ismael Robles Perez}
% Macro con el nombre de la asignatura
\newcommand{\sign}{Seminario de Solución de Problemas de Sistemas Operativos}
% Figura genérica [width]{path, caption}
\newcommand{\fgr}[3][0.8\textwidth]{
    \begin{figure}[h]
        \centering
        \includegraphics[width=#1,keepaspectratio]{#2}
        \caption{#3}
    \end{figure}
}

% Fuente a utilizar en todo el documento


% Header para páginas pares y distintas a cero
\pagestyle{fancy}
\fancyhead{}
\fancyhead[L]{
    \ifthenelse{
        \not \isodd{\value{page}} \and \value{page} > 0} % condición
        % Imprime el título de la actividad a la izquierda del header
        {\actTitle} % si la condición es correcta
        {} % si la condición es falsa
}
% Header para páginas impares
\fancyhead[R]{
    \ifthenelse{
        \isodd{\value{page}}} % condición
        % imprime la asignatura a la derecha del header
        {\sign} % si la condición es correcta
        {} % si la condición es falsa
}

% Footer para páginas impares
\fancyfoot[R]{
    \ifthenelse{
        \isodd{\value{page}}} % condición
        % Imprime mi nombre a la derecha del footer
        {\myName} % si la condición es correcta
        {} % si la condición es falsa
}

% Footer para páginas pares
\fancyfoot[C]{
    \ifthenelse{
        \not \isodd{\value{page}} \and \value{page} > 0} % condición
        % Imprime el número de página al centro del footer
        {\thepage} % si la condición es correcta
        {} % si la condición es falsa
}
        
\renewcommand{\headrulewidth}{0pt}

%%%% Inicia el documento %%%%
\begin{document}

%%%% Portada %%%%
% Nuevo grupo
\begingroup
    % Tamaño de la fuente del grupo
    \fontsize{18pt}{22pt}\selectfont
    \begin{center}
        Universidad de Guadalajara

        Centro Universitario de Ciencias Exactas e Ingenierías \newline

        \includegraphics[width=200px]{imgs/logo_udg.png} \break

        \textbf{\actNum}
        \textbf{\actTitle} \newline
    \end{center}
\endgroup

\begingroup
    \fontsize{16pt}{18pt}\selectfont
        Alumno: \myName

        Código: \code \newline

        Ingeniería en Computación

        \sign

        Calendario: \calendar

        NRC: \nrc

        Sección: \signSec

        \profName \newline \break

        \begin{flushright}
            Guadalajara, Jalisco, México
            
            10 de mayo del 2021
        \end{flushright}
\endgroup


% fin de página (fin de portada)
\newpage

%%%% Contenido %%%%
\blindtext
\blindtext
\blindtext
\newpage

\blindtext
\blindtext
\blindtext
\blindtext
\blindtext
\blindtext
\blindtext

\blindtext
\blindtext

\blindtext

\blindtext
+
Otro blindtext xd

\blindtext

Hola cara de cola

\blindtext

Este es un ejemplo

\blindtext

\end{document}